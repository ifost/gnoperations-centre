%%% Local Variables: 
%%% mode: latex
%%% TeX-master: "gnopc-notes"
%%% End: 
\documentclass[a4paper]{article}

\title{Notes on GnoperationsCentre}

\begin{document}

\maketitle
\newcommand{\gnopc}[0]{ \textsf{Gnopc} }
\begin{itemize}


\section{Goals}

\begin{itemize}
\item To be able to manage OS/390, OS/2,  Netware, Unix, MacOS and MS-Windows
 systems
\item Happy with NAT'ed networks.
\item Can manage mixed IPv6/IPv4 networks, even from IPv4 hosts.
\item Working both ways across a firewall
\item It should provide a mechanism so that VNC takeover of a machine in
 an unreachable network is possible.  Likewise for telnet, ssh and X-windows.
 Maybe also NetBus or BackOrifice.
\item It should provide a logging-of-data mechanism for capacity planning,
and an outage record for helping identify platforms with downtime problems.
\item Safe to use across the greater internet
\item Autodiscovery of networks, routers, services, operating systems
\item To know that some interfaces may be inaccessible to some
protocols because of firewalling
\item Extensible in many languages
\item It should understand clusters,  and the idea that a service may
 ``float'' between several systems.  (A service may be tied to an IP 
 address, not a system)
\item To have web-based, text-based, command-line and GUI-based interfaces
\item If possible,  it should use an existing infrastructure,  or if one
has to be made,  that infrastructure should be one that is useful in some
other context.
\item It should have plain text configuration files that can optionally
be centrally managed.  i.e.  a managed client can decide at a fine level
what functions they want their management company to offer.
\item When some problem occurs,  it should be possible to:
  \begin{itemize}
  \item have the system automatically run some command
  \item have a management server's operator choose a command to have run
  \item page someone,  email someone,  send SMS,  play a sound
  \item see it in a console somewhere
  \end{itemize}
\end{itemize}

\subsection{Monitors}

\begin{itemize}
\item SNMP mib variable
\item Program return code
\item Program output
\item ``Standard thing'' -- cpu load, memory, disk space
\end{itemize}

For each of these \ldots
\begin{itemize}
\item Check a number for a threshold having been exceeded a certain number
 of times,  and going back under threshold
\item Check a string for having a particular pattern
\end{itemize}


\subsection{Network}

\begin{itemize}
\item Ping checks -- is host alive?
\item TCP ping checks -- is service available?  (Also as an alternative
 to a ping check)
\item HTTP and HTTPS checks -- is a web site available?  Does it have
a particular response?
\item SMTP, POP, IMAP checks -- can emails be sent and received?
\item SNMP checks (also SNMP mib variable) -- is a particular interface
 OK, etc?
\end{itemize}

For each of these\ldots
\begin{itemize}
\item Rough sensitivity -- e.g. did we get one OK out of 3 ?
\end{itemize}

\subsection{File monitoring}

\begin{itemize}
\item Read a file for a particular set of patterns
\end{itemize}


\subsection{Databases}

\begin{itemize}
\item Check for upness/down-ness etc
\item Check for standard conditions
\end{itemize}


\section{Scenarios}


\subsection{Monitor Program}

A program is told to perform one of its regular monitoring tests. It
does this, discovers that it is over the appropriate threshold,  and
needs to alert about this.  It sends a message about this to a program
whose jobs it is to care.  Hmm\ldots maybe it actually just lets a 
publish-and-subscribe server know about it.




\subsection{Distant Networks}

Here at Gnopc Management Corp,  we run a \gnopc server in our DMZ (!!??),
which we call \texttt{gnpc.gnopc.com}.   Our staff operate inside our
firewall,  and connect to it through masqueraded connections.  Most of
them just use a web interface,  so their privileges just work off their
login to the web server.  Others use a GUI, so we have to create 
keypairs for them,  and authorisation rules in the various programs.


We have three clients (Easy Corp,  Hard Corp and Bought Corp).  

Easy Corp just want to have hard disk space monitored
on their mail server (\texttt{mail.easycorp.com}),  so it just connects
to port 4973 of \texttt{gnpc.gnopc.com}.  It uses its private key function
so that we know that it really is who it says it is.


Hard Corp have many servers which report back to \texttt{gnpc.hardcorp.com}.
Hard Corp's staff connect to this machine also and get notified about
problems from here.  However,  some of their functions have been outsourced
to us,  so when \gnopc starts up on \texttt{gnpc.hardcorp.com},  it
connects to port 4973 of \texttt{gnpc.gnopc.com} and uses its private
key function.

Bought Corp was bought by Hard Corp,  and has a \gnopc server at
\texttt{b1.boughtcorp.com} which connects to \texttt{gnpc.hardcorp.com}.



\section{Messages}

The complete set of messages is as follows:


\begin{itemize}
\item I have had an event which is of interest to various people.  
 Please forward it.  (Monitor $\rightarrow$ Publishing service)

\item I am interested in hearing about the following events.
 (Mgmt console / event store $\rightarrow$ Publishing service)

\item An event happened which you are interested in.
 (Publishing service $\rightarrow$ Mgmt console / event store)

\item I've just started up.  What interesting events have happened?
 (Mgmt console $\rightarrow$ Event store)

\item Here are the events you wanted to know about 
 (Event store $\rightarrow$ mgmt console)

\item Hello,  are you alive?
 (Anything $\rightarrow$ anything)

\item Yes,  I am alive
 (Anything $\rightarrow$ anything)

\item I have recorded the value X for the attribute Y of data thingy Z.
 (Monitor $\rightarrow$ data store)

\item Please run this command from your list of approved, standard commands
 (Mgmt console $\rightarrow$ Mgmt daemon)

\item Please run this command that you don't know about,  trusting me to
 be good based merely on my identification
 (Mgmt console $\rightarrow$ Mgmt daemon)

\item Please connect to TCP/IP port number P (on interface X.X.X.X),
 Process Q will feed you the data to interact with it.  

\item Hello Process Q,  please listen on TCP/IP port number P2 
 (on interface Y.Y.Y.Y).  Tunnel any data that gets sent to it to
 process Q2 (stream N).

\item Here is some data for that tunnelled IP connection called stream N.


\end{itemize}


\end{document}